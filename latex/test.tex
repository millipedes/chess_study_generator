\documentclass{article}

\usepackage[utf8]{inputenc}
\usepackage[english]{babel}
\usepackage{multicol}
\usepackage{geometry}
\usepackage{comment}
\usepackage{skak}

\geometry{textheight = 22cm}

\newcommand{\n}{$\char`\\ n$}
\newcommand{\tab}{$\quad$}
\newcommand{\smalls}{\par\smallskip}
\newcommand{\meds}{\par\medskip}
\newcommand{\biigs}{\par\bigskip}
\newcommand{\HRule}{\rule{\textwidth}{1pt}}
\newcommand{\csfont}[1]{\fontfamily{cmtt}\selectfont #1}
\newcommand{\XOR}{\mathbin{\char`\^}}
\newcommand{\smalltitle}[1]{\textit{#1}}

\usepackage{fancyhdr}
\pagestyle{fancy}
\fancyhf{}
\fancyhead[L]{Alekhine \hfill \thepage}
\renewcommand{\headrulewidth}{0.4pt}

\markright{Alekhine \hfill}

\begin{document}

\section*{Alekhine Opening}
    \biigs
    \subsection*{Two Pawns Attack}
        \smalls
        \newgame

        \hidemoves{1.e4 Nf6 2. e5 Nd5 3. c4}

        \begin{multicols}{2}
        \showboard
        \smalls
        \begin{tabular}{c|c|c}
        $\#$ & White & Black\\
        \hline
        1 & e4 & Nf6 \\
        2 & e5 & Nd5 \\
        3 & c4 & \\
        \end{tabular}
        \end{multicols}
        \tab Only move after 3. c4 is Nb6, otherwise you lose the knight.  We
        will not consider d4 here because it leads to the four pawns attack and
        the exchange Alekhine.  From here there are two possible moves for white
        4. c5 the chase/Lasker variation or 4. b3 the Steiner Variation.
        \smalls

        \hidemoves{3... Nb6}
        \begin{multicols}{2}
            \showboard
            \smalls
            \begin{tabular}{c|c|c}
                $\#$ & White & Black\\
                \hline
                1 & e4 & Nf6 \\
                2 & e5 & Nd5 \\
                3 & c4 & Nb6 \\
            \end{tabular}
        \end{multicols}


        \storegame{two_pawns_attack}

        \smalls
        \newpage
        \subsubsection*{Steiner Variation}
        \hidemoves{4. b3}

        \begin{multicols}{2}
            \showboard
            \smalls
            \begin{tabular}{c|c|c}
                $\#$ & White & Black\\
                \hline
                1 & e4 & Nf6 \\
                2 & e5 & Nd5 \\
                3 & c4 & Nb6 \\
                4 & b3 & \\
            \end{tabular}
        \end{multicols}

        \tab There is a variation that starts with g6 but it is tough to defend
        for black thus we will focus on d6.

        \hidemoves{4... d6 5. exd6 cxd6 6. Bb2}
        \begin{multicols}{2}
            \showboard
            \smalls
            \begin{tabular}{c|c|c}
            $\#$ & White & Black\\
                \hline
                1 & e4 & Nf6 \\
                2 & e5 & Nd5 \\
                3 & c4 & Nb6 \\
                4 & b3 & d6 \\
                5 & exd6 & cxd6 \\
                6 & Bb2 & \\
            \end{tabular}
        \end{multicols}


        \subsubsection*{Lasker (Chase) Variation}
        \restoregame{two_pawns_attack}
        \hidemoves{4. c5 Nd5}
        \begin{multicols}{2}
            \showboard
            \smalls
            \begin{tabular}{c|c|c}
                $\#$ & White & Black\\
                \hline
                1 & e4 & Nf6 \\
                2 & e5 & Nd5 \\
                3 & c4 & Nb6 \\
                4 & c5 & Nd5 \\
            \end{tabular}
        \end{multicols}
        \storegame{lasker_mainline}

        \subsubsection*{SideLine 5. Bc4}
        \hidemoves{5. Bc4 e6 6. d4 b6 7. cxb6 axb6 8. Nf3 Ba6 9. Bxa6 Rxa6}
        \begin{multicols}{2}
            \showboard
            \smalls
            \begin{tabular}{c|c|c}
                $\#$ & White & Black\\
                \hline
                1 & e4 & Nf6 \\
                2 & e5 & Nd5 \\
                3 & c4 & Nb6 \\
                4 & c5 & Nd5 \\
                5 & Bc4 & e6 \\
                6 & d4 & b6 \\
                7 & cxb6 & axb6 \\
                8 & Nf3 & Ba6 \\
                9 & Bxa6 & Rxa6 \\
            \end{tabular}
        \end{multicols}
        In this variation black's knight on d5 is very annoying.  Eventual plan
        for white is to play Knight c3 and exchange Knights.  If white does
        this, there will be d6 challenging e5 pawn.

        \subsubsection*{Lasker (Chase)  Mainline}
        \restoregame{lasker_mainline}
        \hidemoves{5. Nc3 e6 6. d4 Nxc3 7. bxc3 b6 8. cxb6 axb6 9. Nf3 Bb7 10.
        Bd3 d6 11. O-O}
        \begin{multicols}{2}
            \showboard
            \smalls
            \begin{tabular}{c|c|c}
                $\#$ & White & Black\\
                \hline
                1 & e4 & Nf6 \\
                2 & e5 & Nd5 \\
                3 & c4 & Nb6 \\
                4 & c5 & Nd5 \\
                5 & Nc3 & e6 \\
                6 & d4 & Nxc3 \\
                7 & bxc3 & b6 \\
                8 & cxb6 & axb6 \\
                9 & Nf3 & Bb7 \\
                10 &  Bd3 & d6 \\
                11 & O-O & \\
            \end{tabular}
        \end{multicols}
        If after 5. \dots e6 white plays Nxd5 black has the move d6 to remove
        white's center.

    \subsection*{Four Pawns Attack}
        \newgame
        \hidemoves{1. e4 Nf6 2. e5 Nd5 3. d4 d6 4. c4 Nb6 5. f4}
        \begin{multicols}{2}
            \showboard
            \smalls
            \begin{tabular}{c|c|c}
            $\#$ & White & Black\\
            \hline
            1 & e4 & Nf6 \\
            2 & e5 & Nd5 \\
            3 & d4 & d6 \\
            4 & c4 & Nb6 \\
            5 & f4 & \\
        \end{tabular}
        \end{multicols}
        From this position black can play one of two moves: dxe5 (main line) or
        Bf5 (Trifunovic Variation).

        \storegame{four_pawns_attack}

        \subsubsection*{Trifunovic Variation}
        \hidemoves{5... Bf5 6. Nc3 e6 7. Nf3 dxe5 8. fxe5 Nc6 9. Be3 Be7}
        \begin{multicols}{2}
            \showboard
            \smalls
            \begin{tabular}{c|c|c}
                $\#$ & White & Black\\
                \hline
                1 & e4 & Nf6 \\
                2 & e5 & Nd5 \\
                3 & d4 & d6 \\
                4 & c4 & Nb6 \\
                5 & f4 & Bf5 \\
                6 & Nc3 & e6 \\
                7 & Nf3 & dxe5 \\
                8 & fxe5 & Nc6 \\
                9 & Be3 & Be7 \\
            \end{tabular}
        \end{multicols}
        Basically just a way to potentially confuse your opponent with
        unexpected 5. \dots Bf5, but will transpose to main line.

        \subsubsection*{Interesting Side Line 7. dxc5}
        \restoregame{four_pawns_attack}
        \hidemoves{5... dxe5 6. fxe5 c5 7. dxc5 Qxd1 8. Kxd1 Na4}
        \begin{multicols}{2}
            \showboard
            \smalls
            \begin{tabular}{c|c|c}
                $\#$ & White & Black\\
                \hline
                1 & e4 & Nf6 \\
                2 & e5 & Nd5 \\
                3 & d4 & d6 \\
                4 & c4 & Nb6 \\
                5 & f4 & dxe5 \\
                6 & fxe5 & c5 \\
                7 & dxc5 & Qd1+ \\
                8 & Kxd1 & Na4 \\
            \end{tabular}
        \end{multicols}
        Here black is just better.
        \smalls
        \newpage

        \subsubsection*{Interesting Side Line 7. d5}
        \restoregame{four_pawns_attack}
        \hidemoves{5... dxe5 6. fxe5 c5 7. d5 e6 8. d6 Qh4 9. Ke2 Qe4 10. Kf2
        Qxe5}
        \begin{multicols}{2}
            \showboard
            \smalls
            \begin{tabular}{c|c|c}
                $\#$ & White & Black\\
                \hline
                1 & e4 & Nf6 \\
                2 & e5 & Nd5 \\
                3 & d4 & d6 \\
                4 & c4 & Nb6 \\
                5 & f4 & dxe5 \\
                6 & fxe5 & c5 \\
                7 & d5 & e6 \\
                8 & d6 & Qh4 \\
                9 & Ke2 & Qe4 \\
                10 & Kf2 & Qxe5 \\
            \end{tabular}
        \end{multicols}
        Here black is just better.


        \subsubsection*{Interesting Side Line if 8. Nc3}
        \restoregame{four_pawns_attack}
        \hidemoves{5... dxe5 6. fxe5 c5 7. d5 e6 8. Nc3 exd5 9. cxd5 c4 10. Nf3
        Bb4 11. Bxc4 Bxc3 12. bxc3 Nxc4 13. Qa4 Nd7 14. Qxc4}
        \begin{multicols}{2}
            \showboard
            \smalls
            \begin{tabular}{c|c|c}
                $\#$ & White & Black\\
                \hline
                1 & e4 & Nf6 \\
                2 & e5 & Nd5 \\
                3 & d4 & d6 \\
                4 & c4 & Nb6 \\
                5 & f4 & dxe5 \\
                6 & fxe5 & c5 \\
                7 & d5 & e6 \\
                8 & Nc3 & exd5 \\
                9 & cxd5 & c4 \\
                10 & Nf3 & Bb4 \\
                11 & Bxc3 & Bxc3 \\
                12 & bxc3 & Nxc4 \\
                13 & Qa4 & Nd7 \\
                14 & Qxc4 & \\
            \end{tabular}
        \end{multicols}
        Not the best, and shows opponent knows what to do against 6. \dots c5.
        \smalls
        \newpage

        \subsubsection*{Main Line}
        \restoregame{four_pawns_attack}
        \hidemoves{5... dxe5 6. fxe5 Nc6 7. Be3 Bf5 8. Nc3 e6 9. Nf3 Be7}
        \begin{multicols}{2}
            \showboard
            \smalls
            \begin{tabular}{c|c|c}
                $\#$ & White & Black\\
                \hline
                1 & e4 & Nf6 \\
                2 & e5 & Nd5 \\
                3 & d4 & d6 \\
                4 & c4 & Nb6 \\
                5 & f4 & dxe5 \\
                6 & fxe5 & Nc6 \\
                7 & Be3 & Bf5 \\
                8 & Nc3 & e6 \\
                9 & Nf3 & Be7 \\
            \end{tabular}
        \end{multicols}
        From this position white has two major responses: Be2 (main line) and d5.
        \storegame{four_pawns_attack_main_line}

        \subsubsection*{Main Line 10. d5}
        \hidemoves{10. d5 exd5 11. cxd5 Nb4 12. Nd4 Bd7 13. e6 fxe6 14. dxe6 Bc6
        15. Nxc6 Qxd1 16. Rxd1 Nc2 17. Kd2 Nxe3 18. Kxe3 bxc6}
        \begin{multicols}{2}
            \showboard
            \smalls
            \begin{tabular}{c|c|c}
                $\#$ & White & Black\\
                \hline
                1 & e4 & Nf6 \\
                2 & e5 & Nd5 \\
                3 & d4 & d6 \\
                4 & c4 & Nb6 \\
                5 & f4 & dxe5 \\
                6 & fxe5 & Nc6 \\
                7 & Be3 & Bf5 \\
                8 & Nc3 & e6 \\
                9 & Nf3 & Be7 \\
                10 & d5 & exd5 \\
                11 & cxd5 & Nb4 \\
                12 & Nb4 & Nd7 \\
                13 & e6 & fxe6 \\
                14 & dxe6 & Bc6 \\
                15 & Nxc6 & Qxd1 \\
                16 & Rxd1 & Nc2 \\
                17 & Kd2 & Nxe3 \\
                18 & Kxe3 & bxc6 \\
            \end{tabular}
        \end{multicols}
        Pretty equal position.  e6 is weak for white but black has doubled b
        pawns.
        \smalls
        \newpage

        \subsubsection*{Main Main Line 10. Be2}
        \restoregame{four_pawns_attack_main_line}
        \hidemoves{10. Be2 O-O 11. O-O f6 12. exf6 Bxf6 13. Qd2 Qe7 14. Rad1}
        \begin{multicols}{2}
            \showboard
            \smalls
            \begin{tabular}{c|c|c}
                $\#$ & White & Black\\
                \hline
                1 & e4 & Nf6 \\
                2 & e5 & Nd5 \\
                3 & d4 & d6 \\
                4 & c4 & Nb6 \\
                5 & f4 & dxe5 \\
                6 & fxe5 & Nc6 \\
                7 & Be3 & Bf5 \\
                8 & Nc3 & e6 \\
                9 & Nf3 & Be7 \\
                10 & Be2 & O-O \\
                11 & O-O & f6 \\
                12 & exf6 & Bxf6 \\
                13 & Qd2 & Qe7 \\
                14 & Rad1 & \\
            \end{tabular}
        \end{multicols}
        Here both sides are about equal.  The h8 a1 diagonal is weak for white.
        Black does not want to enter an end game from this position, his pawns
        are a little too weak.

    \subsection*{Balogh Variation}


        \newgame
        \hidemoves{1. e4 Nf6 2. e5 Nd5 3. d4 d6 4. Bc4}
        \begin{multicols}{2}
            \showboard
            \smalls
            \begin{tabular}{c|c|c}
                $\#$ & White & Black\\
                \hline
                1 & e4 & Nf6 \\
                2 & e5 & Nd5 \\
                3 & d4 & d6 \\
                4 & Bc4 & \\
            \end{tabular}
        \end{multicols}

        \subsubsection*{Main Line}
        \hidemoves{4... Nb6}
        \begin{multicols}{2}
            \showboard
            \smalls
            \begin{tabular}{c|c|c}
                $\#$ & White & Black\\
                \hline
                1 & e4 & Nf6 \\
                2 & e5 & Nd5 \\
                3 & d4 & d6 \\
                4 & Bc4 & Nb6 \\
            \end{tabular}
        \end{multicols}
        Here Bd3 is a mistake.

        \storegame{balogh_variation_main_line}

        \subsubsection*{Main Line 5. Bd3}
        \hidemoves{5. Bd3 dxe5 6. dxe5 Nc6 7. Nf3 Bg4 8. Qe2 Bxf3 9. Qxf3 Nxe5
        10. Qxb7 Nxd3 11. cxd3 Qxd3 12. Nc3 e6 13. Qxc7}
        \begin{multicols}{2}
            \showboard
            \smalls
            \begin{tabular}{c|c|c}
                $\#$ & White & Black\\
                \hline
                1 & e4 & Nf6 \\
                2 & e5 & Nd5 \\
                3 & d4 & d6 \\
                4 & Bc4 & Nb6 \\
                5 & Bd3 & dxe5 \\
                6 & dxe5 & Nc6 \\
                7 & Nf3 & Bg4 \\
                8 & Qe2 & Bxf3 \\
                9 & Qxf3 & Nxe5 \\
                10 & Qxb7 & Nxd3 \\
                11 & cxd3 & Qxd3 \\
                12 & Nc3 & e6 \\
                13 & Qxc7 & \\
            \end{tabular}
        \end{multicols}

        \restoregame{balogh_variation_main_line}

        \subsubsection*{Main Line}
        \hidemoves{5. Bb3 dxe5 6. Qh5 e6 7. dxe5 Nc6 8. Nf3 Nd4 9. Nxd4 Qxd4 10.
        O-O g6 11. Qg5 Be7 12. Qg3 Bd7 13. Nc3}
        \begin{multicols}{2}
            \showboard
            \smalls
            \begin{tabular}{c|c|c}
            $\#$ & White & Black\\
            \hline
            1 & e4 & Nf6 \\
            2 & e5 & Nd5 \\
            3 & d4 & d6 \\
            4 & Bc4 & Nb6 \\
            5 & Bb3 & dxe5 \\
            6 & Qh5 & e6 \\
            7 & dxe5 & Nc6 \\
            8 & Nf3 & Nd4 \\
            9 & Nxd4 & Qxd4 \\
            10 & O-O & g6 \\
            11 & Qg5 & Be7 \\
            12 & Qg3 & Bd7 \\
            13 & Nc3 & \\
        \end{tabular}
        \end{multicols}
        \smalls
        \newpage

    \subsection*{Exchange Variation}

        \subsubsection*{Main Line}
        \newgame
        \hidemoves{1. e4 Nf6 2. e5 Nd5 3. d4 d6 4. c4 Nb6 5. exd6 cxd6 6. Nc3 g6
        7. Be3 Bg7 8. Rc1 O-O 9. b3 e5 10. dxe5 dxe5 11. Qxd8 Rxd8 12. c5 N6d7
        13. Bc4 Nc6 14. Nf3 h6}
        \begin{multicols}{2}
            \showboard
            \smalls
            \begin{tabular}{c|c|c}
            $\#$ & White & Black\\
            \hline
            1 & e4 & Nf6 \\
            2 & e5 & Nd5 \\
            3 & d4 & d6 \\
            4 & c4 & Nb6 \\
            5 & exd6 & cxd6 \\
            6 & Nc3 & g6 \\
            7 & Be3 & Bg7 \\
            8 & Rc1 & O-O \\
            9 & b3 & e5 \\
            10 & dxe5 & dxe5 \\
            11 & Qxd8 & Rxd8 \\
            12 & c5 & N6d7 \\
            13 & Bc4 & Nc6 \\
            14 & Nf3 & h6 \\
        \end{tabular}
        \end{multicols}

        \subsection*{Modern Variation}

        \subsubsection*{Main Line}
        \newgame
        \hidemoves{1. e4 Nf6 2. e5 Nd5 3. d4 d6 4. Nf3 Bg4 5. Be2 e6 6. O-O Be7
        7. c4 Nb6 8. Nc3 O-O 9. exd6 cxd6}
        \begin{multicols}{2}
            \showboard
            \smalls
            \begin{tabular}{c|c|c}
            $\#$ & White & Black\\
            \hline
            1 & e4 & Nf6 \\
            2 & e5 & Nd5 \\
            3 & d4 & d6 \\
            4 & c4 & Nf3 \\
            5 & Be2 & e6 \\
            6 & O-O & Be7 \\
            7 & c4 & Nb6 \\
            8 & Nc3 & O-O \\
            9 & exd6 & cxd6 \\
        \end{tabular}
        \end{multicols}

\end{document}
